% MỞ ĐẦU (LỜI MỞ ĐẦU - Chương 0)

\chapter*{MỞ ĐẦU} % Tên của chương
\addcontentsline{toc}{chapter}{MỞ ĐẦU} % Thêm tên chương vào mục lục

\label{Chapter0} % Để trích dẫn chương này ở chỗ nào đó trong bài, hãy sử dụng lệnh \ref{Chapter0} 

Hiện nay, Internet là một thứ đã rất phổ biến trên toàn thế giới. Những lợi ích của nó mang lại là không thể chối cãi: giải trí, cập nhật thông tin nhanh chóng, kết nối mọi người với nhau bất kể khoảng cách,.... Từ đó Internet of Thing (IoT), một công nghệ với cốt lõi là các tính năng của Internet, đã được ra đời. Với IoT, những vật dụng tưởng chừng vô tri như bóng đèn, quạt điện, tủ lạnh, xe máy, công tơ điện,... lại có thể giao tiếp, truyền tải thông tin với nhau. Để làm được điều đó thì không thể thiếu được các giao thức, thứ được ví như là các quy tắc giao tiếp chung trong thế giới IoT. 

Trên thực tế, đã có rất nhiều hệ thống IoT được thương mại hóa, áp dụng trong tất cả các lĩnh vực của đời sống. Tất cả các hệ thống này đều có một yêu cầu chung là truyền tải thông tin nhanh chóng, an toàn và ít xảy ra lỗi.

Xuất phát từ yêu cầu đó, đề tài "\textbf{Giao thức MQTT và các ứng dụng trong IoT}" đã được tôi lựa chọn. Tiểu luận này sẽ nghiên cứu giao thức MQTT, một giao thức được phát triển với mục đích truyền tải các thông tin nhanh chóng, có thể sử dụng kể cả khi kết nối mạng không ổn định. Dựa trên cơ sở đó áp dụng vào mô hình trao đổi dữ liệu nhiệt độ độ ẩm từ cảm biến DHT11 thông qua hai board điều khiển ESP8266.

Các phần chính của tiểu luận này bao gồm những phần sau:
\begin{itemize}
	\item {\textbf{Chương 1 - Tổng quan về Internet:}}
	Trong chương này sẽ trình bày các kiến thức liên quan tới Internet bao gồm lịch sử phát triển,các thành phần chính và cách vận hành.
	\item {\textbf{Chương 2 - Tổng quan về IoT và giao thức MQTT:}}
	Chương này trình bày tổng quan về IoT, các thành phần của nó. Tiếp đến là giới thiệu về giao thức MQTT, lí do nó được chọn là giao thức nghiên cứu trong tiểu luận lần này.
	\item {\textbf{Chương 3 - Thực nghiệm:}}
	Thông tin phần cứng, phần mềm được áp dụng sẽ được trình bày ở chương này. Cùng với đó là cách thiết kế hệ thống, sơ đồ hoạt động, thuật toán phần mềm.
	\item {\textbf{Kết luận:}}
	Cuối cùng là tổng kết lại kết quả thu được sau một thời gian nghiên cứu đề tài, những ứng dụng, hạn chế của mô hình và phương hướng giải quyết.
\end{itemize}