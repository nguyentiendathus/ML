\chapter{MỞ ĐẦU} % Tên của chương

\label{Chapter1} % Để trích dẫn chương này ở chỗ nào đó trong bài, hãy sử dụng lệnh \ref{Chapter1} 

%----------------------------------------------------------------------------------------

% Định nghĩa một số lệnh cần thiết để điều chỉnh định dạng cho một số nội dung nhất định trong bài
\newcommand{\keyword}[1]{\textbf{#1}}
\newcommand{\tabhead}[1]{\textbf{#1}}
\newcommand{\code}[1]{\texttt{#1}}
\newcommand{\file}[1]{\texttt{\bfseries#1}}
\newcommand{\option}[1]{\texttt{\itshape#1}}

%----------------------------------------------------------------------------------------
Hiện nay trí tuệ nhân tạo là một xu hướng công nghệ trên toàn thế giới. Nó đã xuất hiện ở mọi nơi như là mạng xã hội, thiết bị điện tử, xe cộ, .... Xuất phát từ nhu cầu ngày càng cao của con người về sự tiện lợi, thông tin nhanh chóng thì mô hình nhận diện hành động con người cũng đã được ra đời.

Nhận diện hành động con người ở ảnh tĩnh có rất nhiều tiềm năng ứng dụng vào đời sống như là thiết bị hỗ trợ người khiếm thị, gán nhãn tự động, .... Một giải pháp trực tiếp cho vấn đề đó chính là từ hình ảnh sinh ra được văn bản mô tả chính xác ảnh đó với lời văn hợp lý. Tuy nhiên thử thách được đặt ra đó là con người có rất nhiều trạng thái hoạt động phức tạp khác nhau cùng với ngữ pháp mỗi loại ngôn ngữ là khác nhau. Chính vì vậy bài báo cáo này sẽ tập trung vào một số trạng thái cơ bản của con người và mô tả nó bằng tiếng Anh. Các mô hình Machine learning và Deep learning được sử dụng trong bài báo cáo này là CNN, RNN và LSTM. Đây là các mô hình rất phổ biến trên thế giới, được tin dùng trong việc xử lí ảnh và ngôn ngữ tự nhiên.