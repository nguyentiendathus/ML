%----------------------------------------------------------------------------------------
%	(KHÔNG CHỈNH SỬA PHẦN NÀY)
%
%	PHẦN 1: CÁC PACKAGE CƠ BẢN VÀ CÁC TÙY CHỈNH VĂN BẢN
%----------------------------------------------------------------------------------------

\documentclass[
12pt,
oneside,
english,
doublespacing,
nolistspacing,
liststotoc,
parskip,
headsepline,
chapterinoneline,
]{MastersDoctoralThesis}

% \usepackage[utf8]{inputenc} 
\usepackage[utf8]{vietnam} 
%\usepackage[T1]{fontenc}

\usepackage{mathptmx}
\usepackage{amsmath}
\usepackage{textcomp}
\allowdisplaybreaks

%https://www.overleaf.com/learn/latex/Biblatex_citation_styles
%\usepackage[backend=bibtex,style=authoryear,natbib=true]{biblatex}
\usepackage[backend=bibtex,style=numeric,citestyle=ieee,natbib=true]{biblatex}

\addbibresource{main.bib}

\usepackage[autostyle=true]{csquotes}


%----------------------------------------------------------------------------------------
%	PHẦN 2: CÁC PACKAGE BỔ TRỢ THÊM VÀO TRONG QUÁ TRÌNH BIÊN SOẠN
%----------------------------------------------------------------------------------------

\RequirePackage{setlst}		% Liệt kê/trích dẫn code

\usepackage{multirow}
\usepackage{subfigure}
\usepackage[fontsize=13pt]{scrextend}

%https://www.sascha-frank.com/latex-font-size.html
%https://tex.stackexchange.com/questions/103286/how-to-change-section-subsection-font-size
\usepackage{titlesec}

\titleformat{\section}
{\normalfont\fontsize{13}{13}\bfseries}{\thesection}{1em}{}

\titleformat{\subsection}
{\normalfont\fontsize{13}{13}\bfseries\itshape}{\thesubsection}{1em}{}

%https://tex.stackexchange.com/questions/351961/how-to-indent-code-in-beginverbatim
\usepackage{fancyvrb} 		% Fancy Verbatim
\fvset{tabsize=4,vspace=0pt,fontsize=\footnotesize}

\usepackage{longtable} 		% Bảng dài - Long table

\usepackage[figuresright]{rotating} % Bảng ngang - Sideways table
\usepackage{tabularx}

\usepackage{fontawesome5} 	% Các biểu tượng, ký hiệu đặc biệt

\usepackage{tikz} 			% Vẽ hình
\usepackage{indentfirst}
\setlength{\parindent}{0.5cm}	%Thut dau dong cho doan van
\setlength{\parskip}{1.4ex plus 0.5ex minus 0.3ex}		% Tạo khoảng trống giữa 2 đoạn văn %spacing between two paragraph

\usetikzlibrary{calc}
